\definecolor{cardLight}{HTML}{C9DAF8} % Mã #C9DAF8 (Card 1 và 3)
\definecolor{cardDark}{HTML}{B7CFF4}  % Mã #B7CFF4 (Card 2)

%---------------------------------------------------------
%	PRESENTATION BODY SLIDES
%---------------------------------------------------------
\section{Đặt vấn đề \& mục tiêu} % Note all sections and subsections are automatically placed in your table of contents
\SectionIntro{Phần 1: Đặt vấn đề \& tổng quan}{Bối cảnh năng lượng tái tạo và thách thức trong công tác bảo trì.}

%------------------------------------------------
% \begin{frame}
%     \frametitle{BỐI CẢNH VÀ TÍNH CẤP THIẾT}
%     \vspace{-0.2cm}
%     % Fix overfull hbox by forcing exact bounding box
%     \setlength{\emergencystretch}{3em}
%     \begin{columns}[c,onlytextwidth]
%         \begin{column}{0.42\textwidth}
%             \justifying
%             {\scriptsize Năng lượng mặt trời (PV) đang bùng nổ toàn cầu với tốc độ tăng trưởng \textbf{>20\%} mỗi năm (theo IEA). Tại Việt Nam, xu hướng này đang phát triển mạnh mẽ. Tính đến đầu năm 2025, điện mặt trời đã chiếm \textbf{11.85\%} tổng công suất nguồn điện cả nước.}

%             \vspace{0.05cm}

%             % Example box
%             \begin{tikzpicture}
%                 \node[inner sep=3pt,rounded corners=0pt] (box) {
%                     % Reduced width to 0.92 to ensure padding (inner sep) doesn't cause overflow
%                     \begin{minipage}{0.92\linewidth}
%                         {\scriptsize \hspace{0.1cm}\textbf{Ví dụ: NM điện mặt trời Dầu Tiếng}}
%                         {\scriptsize
%                         \begin{itemize}\setlength\itemsep{0pt}\setlength\parskip{0pt}
%                             \item[\textcolor{green!50!black}{\ding{51}}] \textbf{Công suất:} >600 MWp.
%                             \item[\textcolor{green!50!black}{\ding{51}}] \textbf{Quy mô:} Hàng trăm héc-ta.
%                             \item[\textcolor{green!50!black}{\ding{51}}] \textbf{Thách thức:} Giám sát thủ công khó khăn.
%                         \end{itemize}}
%                     \end{minipage}
%                 };
%                 \begin{scope}[on background layer]
%                     \fill[gray!15,rounded corners=0pt] ($(box.north west)+(-0.04cm,0.04cm)$) rectangle ($(box.south east)+(0.04cm,-0.04cm)$);
%                 \end{scope}
%             \end{tikzpicture}

%             \vspace{0.05cm}

%             \justifying
%             {\scriptsize Theo \textbf{Quy hoạch điện VIII}, dư địa phát triển năng lượng tái tạo vẫn lớn, dự kiến điện mặt trời tiếp tục \textbf{tăng trưởng mạnh đến 2030.}}
%         \end{column}%
%         \begin{column}{0.52\textwidth}
%             \begin{center}
%                 \begin{tikzpicture}
%                     % Use 0.99\linewidth to avoid rounding errors causing overfull hbox
%                     \pgfmathsetmacro{\w}{0.99*\linewidth}
%                     \pgfmathsetmacro{\h}{\w * 0.6} % Aspect ratio
                    
%                     % FORCE Bounding box to be exactly the linewidth
%                     % This hides the 0.75pt border spill from TeX's layout engine
%                     \useasboundingbox (0,0) rectangle (\w pt, \h pt);

%                     \begin{scope}
%                         \clip[rounded corners=8pt] (0,0) rectangle (\w pt, \h pt);
%                         \node[anchor=center,inner sep=0pt] at (0.5*\w pt, 0.5*\h pt) {\includegraphics[width=\w pt, height=\h pt, keepaspectratio=false]{images/DatVanDe/dautieng.jpg}};
%                     \end{scope}
%                     % Draw the border on top
%                     \draw[line width=1.5pt,rounded corners=8pt] (0,0) rectangle (\w pt, \h pt);
%                 \end{tikzpicture}
%             \end{center}
%         \end{column}
%     \end{columns}

% \end{frame}

%------------------------------------------------

\begin{frame}
\label{Bungno}
	\frametitle{TÌNH HÌNH HỆ THỐNG PIN MẶT TRỜI TRÊN THẾ GIỚI}
    
    \vspace{-0.2cm}
    
    % --- PHẦN 1: TỔNG QUAN (Bullet points) ---
    \begin{itemize}
        \setlength\itemsep{0.2em} % Giảm khoảng cách dòng
        \small % Tăng size chữ lên small
        \item Năng lượng tái tạo đang được ưu tiên phát triển trên toàn cầu.
        \item Điện mặt trời chiếm tỷ trọng lớn nhất trong các nguồn NLTT.
        \item Công suất lắp đặt tăng nhanh, tập trung ở các quốc gia lớn.
        \item Mô hình triển khai chủ yếu: hệ thống quang điện (PV).
    \end{itemize}

    \vspace{0.3cm}

    % --- PHẦN 2: BIỂU ĐỒ VÀ BẢNG SỐ LIỆU ---
    \begin{columns}[c, onlytextwidth]
        
        % Cột trái: Biểu đồ
        \begin{column}{0.48\textwidth}
            \centering
            % Ảnh width = 0.9\linewidth
            \includegraphics[width=0.9\linewidth]{images/TongQuan/bieudobungno.png}
        \end{column}

        % Cột phải: Bảng số liệu
        \begin{column}{0.48\textwidth}
            \centering
            % Resize bảng width = 0.9\linewidth (bằng ảnh)
            % Tăng tabcolsep lên cao để bảng bè ra (tăng chiều rộng tự nhiên),
            % giúp khi resize về cùng width sẽ giảm chiều cao xuống cho bằng ảnh.
            \resizebox{0.9\linewidth}{!}{
                \setlength{\tabcolsep}{12pt} % Tăng khoảng cách cột
                \renewcommand{\arraystretch}{0.85} % Giảm khoảng cách dòng
                \begin{tabular}{|c|l|r|}
                    \hline
                    \textbf{STT} & \textbf{Quốc gia} & \textbf{Công suất (MW)} \\
                    \hline
                    1 & Trung Quốc & 887,360 \\
                    2 & Hoa Kỳ & 175,990 \\
                    3 & Ấn Độ & 97,042 \\
                    4 & Nhật Bản & 91,610 \\
                    5 & Đức & 89,943 \\
                    6 & Brazil & 53,113 \\
                    7 & Úc & 38,469 \\
                    8 & Tây Ban Nha & 36,285 \\
                    9 & Italy & 36,008 \\
                    10 & Korea Rep & 26,445 \\
                    \hline
                \end{tabular}
            }
        \end{column}

    \end{columns}
\end{frame}

\begin{frame}
    \frametitle{TÌNH HÌNH HỆ THỐNG PIN MẶT TRỜI TRÊN VIỆT NAM}
    
    \vspace{0cm}
    
    % --- PHẦN 1: NỘI DUNG CHÍNH ---
    \begin{itemize}
        \setlength\itemsep{0.1em}
        \footnotesize
        \item Việt Nam ưu tiên phát triển năng lượng tái tạo.
        \item Định hướng rõ ràng từ Quy hoạch điện VII.
        \item Nhiều trang trại điện mặt trời quy mô lớn đã đưa vào vận hành chủ yếu ở Nam Trung Bộ, Tây Nguyên và Nam Bộ.
    \end{itemize}

    \vspace{0.1cm}

    % --- PHẦN 2: HÌNH ẢNH MINH HỌA (Lưới 2x2) ---
    \begin{columns}[c, onlytextwidth]
        % Cột 1
        \begin{column}{0.48\textwidth}
            \centering
            \def\imgwidth{0.95\linewidth}
            % Ảnh 1: Phú Yên
            \begin{tikzpicture}
                \clip[rounded corners=5pt] (0,0) rectangle (\imgwidth, 1.7cm);
                \node[anchor=south west, inner sep=0] at (0,0) {\includegraphics[height=1.7cm, width=\imgwidth]{images/DatVanDe/pin/phuyen.jpg}};
            \end{tikzpicture} \\
            {\tiny \textbf{Hòa Hội (Phú Yên)}}
            
            \vspace{0.1cm}
            
            % Ảnh 3: Ninh Thuận
            \begin{tikzpicture}
                \clip[rounded corners=5pt] (0,0) rectangle (\imgwidth, 1.7cm);
                \node[anchor=south west, inner sep=0] at (0,0) {\includegraphics[height=1.7cm, width=\imgwidth]{images/DatVanDe/pin/ninhthuan.jpg}};
            \end{tikzpicture} \\
            {\tiny \textbf{Ninh Phước (Ninh Thuận)}}
        \end{column}

        % Cột 2
        \begin{column}{0.48\textwidth}
            \centering
            \def\imgwidth{0.95\linewidth}
            % Ảnh 2: Dầu Tiếng
            \begin{tikzpicture}
                \clip[rounded corners=5pt] (0,0) rectangle (\imgwidth, 1.7cm);
                \node[anchor=south west, inner sep=0] at (0,0) {\includegraphics[height=1.7cm, width=\imgwidth]{images/DatVanDe/pin/dautieng.jpg}};
            \end{tikzpicture} \\
            {\tiny \textbf{Dầu Tiếng (Tây Ninh)}}
            
            \vspace{0.1cm}
            
            % Ảnh 4: Trung Nam
            \begin{tikzpicture}
                \clip[rounded corners=5pt] (0,0) rectangle (\imgwidth, 1.7cm);
                \node[anchor=south west, inner sep=0] at (0,0) {\includegraphics[height=1.7cm, width=\imgwidth]{images/DatVanDe/pin/trungnam.jpg}};
            \end{tikzpicture} \\
            {\tiny \textbf{Trung Nam (Ninh Thuận)}}
        \end{column}
    \end{columns}
\end{frame}
%------------------------------------------------
\begin{frame}
    \frametitle{THÁCH THỨC}
    \vspace{0.2cm}
    
    \begin{columns}[c, onlytextwidth]
        % --- CỘT TRÁI: HÌNH ẢNH MINH HỌA ---
        \begin{column}{0.45\textwidth}
            \centering
            \begin{tikzpicture}
                % Khung bo góc tròn, chiều cao khoảng 5cm để cân đối với nội dung bên phải
                \clip [rounded corners=12pt] (0,0) rectangle (\linewidth, 5.5cm);
                % Chèn ảnh, fill đầy khung
                % Tăng width lên 1.7 và inner sep=0 để đảm bảo ảnh lấp đầy toàn bộ khung clip
                \node[anchor=center, inner sep=0] at (0.5\linewidth, 2.75cm) {
                    \includegraphics[width=1.7\linewidth]{images/TongQuan/mat.jpg}
                };
            \end{tikzpicture}
        \end{column}

        % --- CỘT PHẢI: NỘI DUNG ---
        \begin{column}{0.5\textwidth}
            % Dùng minipage với chiều cao cố định 5.5cm (bằng chiều cao ảnh bên trái)
            % để căn đều: Đầu itemize trùng đầu ảnh, Đáy box trùng đáy ảnh
            \begin{minipage}[c][5.5cm][s]{\linewidth}
                \begin{itemize}
                    \setlength\itemsep{0.4cm} 
                    \raggedright
                    \item Tốn kém quá nhiều thời gian và nhân lực.
                    \item Nguy cơ mất an toàn lao động cho kỹ thuật viên.
                    \item Hiệu quả không cao.
                \end{itemize}

                \vfill % Đẩy box xuống đáy để căn đáy với ảnh

                % Box Mục tiêu
                \begin{tcolorbox}[
                    enhanced,
                    colback=gray!10,
                    colframe=gray!40,
                    arc=8pt,
                    boxrule=1pt,
                    left=6pt, right=6pt, top=6pt, bottom=6pt,
                    width=\linewidth
                ]
                    \small
                    \textbf{Mục tiêu:} tạo ra một giải pháp cân bằng giữa độ chính xác và chi phí, giúp phát hiện sớm hư hỏng để tối ưu hóa vận hành.
                \end{tcolorbox}
            \end{minipage}
        \end{column}
    \end{columns}
    
\end{frame}

% --- ĐỊNH NGHĨA MÀU (Trích xuất từ ảnh) ---
% Màu Xanh (Box trái)
\definecolor{myBlueBorder}{HTML}{1FA2F2} % Màu thanh bên trái
\definecolor{myBlueBG}{HTML}{F0F8FF}     % Màu nền xanh nhạt

% Màu Cam (Box phải)
\definecolor{myOrangeBorder}{HTML}{F7A623} % Màu thanh bên trái
\definecolor{myOrangeBG}{HTML}{FFFBF2}     % Màu nền cam nhạt

% --- ĐỊNH NGHĨA KHUNG (Custom Box) ---
\newtcolorbox{objectivebox}[2]{
    enhanced,
    equal height group=obj,
    boxrule=0pt, frame hidden, % Ẩn viền xung quanh
    borderline west={0pt}{0pt}{#1}, % Chỉ kẻ vạch màu bên trái dày 4pt
    colback=#2, % Màu nền
    sharp corners, % Góc vuông (hoặc chỉnh arc=3pt nếu muốn bo nhẹ)
    left=5pt, right=5pt, top=10pt, bottom=10pt, % Căn lề trong
    valign=center,
    halign=center
}



\begin{frame}
    \frametitle{MỤC TIÊU NGHIÊN CỨU}

    \begin{columns}[c, onlytextwidth] % Căn lề giữa
        % --- CỘT TRÁI (Xử lý ảnh) ---
        \begin{column}{0.48\textwidth}
            \begin{objectivebox}{myBlueBorder}{myBlueBG}
                \centering
                % Đường dẫn ảnh đã được đổi dấu \ thành / để LaTeX hiểu
                \includegraphics[width=1.5cm]{images/DatVanDe/icons/xulyanh.png} 
                
                \vspace{0.5cm}
                
                \raggedright % Căn trái văn bản
                Xây dựng phương pháp \textbf{xử lý ảnh} để tự động tách chiết từng tấm pin từ ảnh toàn cảnh chụp bởi UAV.
            \end{objectivebox}
        \end{column}

        % --- CỘT PHẢI (Học sâu) ---
        \begin{column}{0.48\textwidth}
            \begin{objectivebox}{myOrangeBorder}{myOrangeBG}
                \centering
                \includegraphics[width=1.5cm]{images/DatVanDe/icons/hocsau.png}
                
                \vspace{0.5cm}
                
                \raggedright
                Phát triển mô hình \textbf{Học sâu (Deep Learning)} để phân loại trạng thái tấm pin (Bình thường vs. Lỗi) với độ chính xác cao.
            \end{objectivebox}
        \end{column}
    \end{columns}
\end{frame}

\begin{frame}
    \frametitle{PHẠM VI NGHIÊN CỨU}

    % Điều chỉnh khoảng cách trên (0cm để vừa vặn, không quá sát title)
    \vspace{0.0cm}
    
    % --- ROW 1: Dữ liệu ---
    \begin{columns}[c] 
        \begin{column}{0.65\textwidth}
            \begin{itemize}
                \setlength\itemsep{0pt}
                \item \textbf{Dữ liệu nghiên cứu:} \\ 
                Ảnh màu (RGB) độ phân giải cao, thu thập thực tế từ UAV tại khu vực khảo sát.
            \end{itemize}
        \end{column}
        \begin{column}{0.33\textwidth}
            \centering
            % Giảm chiều cao ảnh xuống 1.8cm
            \begin{tikzpicture}
                \clip[rounded corners=8pt] (0,0) rectangle (3.6cm, 1.8cm);
                \node[anchor=center, inner sep=0pt] at (1.8cm, 0.9cm) {
                    \includegraphics[width=3.6cm, height=1.8cm, keepaspectratio=false]{images/DatVanDe/minhhoa.jpg}
                };
            \end{tikzpicture}
        \end{column}
    \end{columns}
    
    \vspace{0.2cm} % Giảm khoảng cách giữa các hàng cho vừa trang

    % --- ROW 2: Đối tượng ---
    \begin{columns}[c]
        \begin{column}{0.65\textwidth}
            \begin{itemize}
                \setlength\itemsep{0pt}
                \item \textbf{Đối tượng áp dụng:} \\ 
                Hệ thống pin mặt trời tại các trang trại năng lượng mặt trời quy mô lớn.
            \end{itemize}
        \end{column}
        \begin{column}{0.33\textwidth}
            \centering
            \begin{tikzpicture}
                \clip[rounded corners=8pt] (0,0) rectangle (3.6cm, 1.8cm);
                \node[anchor=center, inner sep=0pt] at (1.8cm, 0.9cm) {
                    \includegraphics[width=3.6cm, height=1.8cm, keepaspectratio=false]{images/DatVanDe/canhdong.jpg}
                };
            \end{tikzpicture}
        \end{column}
    \end{columns}
    
    \vspace{0.2cm}

    % --- ROW 3: Giới hạn ---
    \begin{columns}[c]
        \begin{column}{0.65\textwidth}
            \begin{itemize}
                \setlength\itemsep{0pt}
                \item \textbf{Giới hạn đề tài:} \\ 
                Chỉ phát hiện các hư hỏng vật lý quan sát được trên bề mặt tấm pin; không xét đến các đo đạc chuyên sâu về đặc tính điện.
            \end{itemize}
        \end{column}
        \begin{column}{0.33\textwidth}
            \centering
            \begin{tikzpicture}
                \clip[rounded corners=8pt] (0,0) rectangle (3.6cm, 1.8cm);
                \node[anchor=center, inner sep=0pt] at (1.8cm, 0.9cm) {
                    \includegraphics[width=3.6cm, height=1.8cm, keepaspectratio=false]{images/DatVanDe/ban.jpg}
                };
            \end{tikzpicture}
        \end{column}
    \end{columns}

\end{frame}
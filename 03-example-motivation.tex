\definecolor{cardLight}{HTML}{C9DAF8} % Mã #C9DAF8 (Card 1 và 3)
\definecolor{cardDark}{HTML}{B7CFF4}  % Mã #B7CFF4 (Card 2)

%---------------------------------------------------------
%	PRESENTATION BODY SLIDES
%---------------------------------------------------------
\section{Đặt vấn đề} % Note all sections and subsections are automatically placed in your table of contents

%------------------------------------------------
\begin{frame}
    \frametitle{MỤC TIÊU \& GIẢI PHÁP}
    
    % Image at the top
    \begin{figure}[h!]
        \centering
        \includegraphics[width=0.45\textwidth]{images/DatVanDe/imagetarget.png}
    \end{figure}
    
    \vspace{-0.4cm}
    
    % Solution title
    \begin{center}
        \small Giải pháp: UAV (Flycam) + AI (EfficientNet-B2)
    \end{center}
    
    \vspace{0.2cm}
    
    % Statistics
    \begin{center}
        \begin{tabular}{c@{\hspace{1cm}}c@{\hspace{1cm}}c}
            {\huge \bfseries \textcolor{blue!80!cyan}{96.45\%}} & 
            \textcolor{gray}{\rule[-0.5ex]{1pt}{3ex}} & 
            {\huge \bfseries \textcolor{blue!80!cyan}{96.9\%}} \\[0.3ex]
            \small Độ chính xác & & \small Tách chiếc pin
        \end{tabular}
    \end{center}

\end{frame}

%------------------------------------------------
\begin{frame}
    \frametitle{BỐI CẢNH VÀ TÍNH CẤP THIẾT}
    \vspace{-0.2cm}
    \begin{columns}[T,onlytextwidth]
        \begin{column}{0.52\textwidth}
            \justifying
            {\scriptsize Năng lượng mặt trời (PV) đang bùng nổ toàn cầu với tốc độ tăng trưởng \textbf{>20\%} mỗi năm (theo IEA). Tại Việt Nam, xu hướng này đang phát triển mạnh mẽ. Tính đến đầu năm 2025, điện mặt trời đã chiếm \textbf{11.85\%} tổng công suất nguồn điện cả nước, hàng loạt dự án quy mô lớn đã đi vào vận hành ổn định.}

            \vspace{0.1cm}

            % Example box with left orange strip
            \begin{tikzpicture}
                \node[inner sep=3pt,rounded corners=0pt] (box) {
                    \begin{minipage}{0.94\linewidth}
                        {\scriptsize \hspace{0.3cm}\textbf{Ví dụ điển hình: Nhà máy điện mặt trời Dầu Tiếng}}\\[2pt]
                        {\scriptsize
                        \begin{itemize}\setlength\itemsep{0pt}\setlength\parskip{0pt}
                            \item[\textcolor{green!50!black}{\ding{51}}] \textbf{Công suất:} >600 MWp (Lớn nhất Đông Nam Á).
                            \item[\textcolor{green!50!black}{\ding{51}}] \textbf{Quy mô:} Hàng trăm héc-ta với hàng triệu tấm pin.
                            \item[\textcolor{green!50!black}{\ding{51}}] \textbf{Thách thức:} Giám sát thủ công là bất khả thi.
                        \end{itemize}}
                    \end{minipage}
                };
                % background rounded box and light gray fill
                \begin{scope}[on background layer]
                    \fill[gray!15,rounded corners=0pt] ($(box.north west)+(-0.04cm,0.04cm)$) rectangle ($(box.south east)+(0.04cm,-0.04cm)$);
                    % orange left strip
                    \fill[orange] ($(box.north west)+(-0.04cm,0.04cm)$) rectangle ($(box.south west)+(0.12cm,-0.04cm)$);
                \end{scope}
            \end{tikzpicture}

            \vspace{0.1cm}

            \justifying
            {\scriptsize Theo \textbf{Quy hoạch điện VIII}, dư địa phát triển năng lượng tái tạo vẫn còn rất lớn, dự kiến quy mô điện mặt trời sẽ tiếp tục \textbf{tăng trưởng mạnh mẽ đến năm 2030.}}
        \end{column}

        \begin{column}{0.48\textwidth}
            \begin{center}
                \vspace{0.8cm}
                \begin{tikzpicture}
                    % Draw the clipped image
                    \begin{scope}
                        \clip[rounded corners=10pt] (0,0) rectangle (5.2,3.5);
                        \node[anchor=south west,inner sep=0pt] at (0,0) {\includegraphics[width=5.2cm,keepaspectratio]{images/DatVanDe/dautieng.jpg}};
                    \end{scope}
                    % Draw the border on top
                    \draw[line width=1.5pt,rounded corners=10pt] (0,0) rectangle (5.2,3.5);
                \end{tikzpicture}
            \end{center}
        \end{column}
    \end{columns}

\end{frame}

%------------------------------------------------
\begin{frame}
    \frametitle{THÁCH THỨC}
    \vspace{-0.5cm}
    
    % Three challenge cards
    \begin{center}
    \begin{tikzpicture}[scale=0.85, every node/.style={scale=0.85}]
        % Card 1 - Time and Labor Cost (Dùng màu cardLight)
        \node[rounded corners=12pt, fill=cardLight, minimum width=3.8cm, minimum height=4.2cm, align=center] at (0,0) (card1) {};
        
        % Icon 1 - Time and labor
        \node at (0,0.7) {\includegraphics[width=2cm]{images/DatVanDe/icons/thoigian.png}};
        
        \node[text width=3.5cm, align=center, font=\small\bfseries] at (0,-0.95) {Tốn kém quá\\nhiều thời gian và\\nhân lực};
        
        % Card 2 - Safety Risk (Dùng màu cardDark)
        \node[rounded corners=12pt, fill=cardDark, minimum width=3.8cm, minimum height=4.2cm, align=center] at (4.8,0) (card2) {};
        
        % Icon 2 - Safety
        \node at (4.8,0.7) {\includegraphics[width=2cm]{images/DatVanDe/icons/antoan.png}};
        
        \node[text width=3.5cm, align=center, font=\small\bfseries] at (4.8,-0.95) {Nguy cơ mất\\an toàn lao động\\cho kỹ thuật viên};
        
        % Card 3 - Low Efficiency (Dùng màu cardLight)
        \node[rounded corners=12pt, fill=cardLight, minimum width=3.8cm, minimum height=4.2cm, align=center] at (9.6,0) (card3) {};
        
        % Icon 3 - Efficiency
        \node at (9.6,0.7) {\includegraphics[width=2cm]{images/DatVanDe/icons/hieuqua.png}};
        
        \node[text width=3.5cm, align=center, font=\small\bfseries] at (9.6,-0.95) {Hiệu quả\\không cao};
    \end{tikzpicture}
    \end{center}
    
    \vspace{0.15cm}
    
    % Objective box
    \begin{center}
    \begin{tikzpicture}
        \node[inner sep=6pt, text width=0.88\textwidth] (objbox) {
            \small \textbf{Mục tiêu:} tạo ra một giải pháp cân bằng giữa độ chính xác và chi phí, giúp phát hiện sớm hư hỏng để tối ưu hóa vận hành
        };
        \begin{scope}[on background layer]
            \fill[gray!8, rounded corners=4pt] (objbox.north west) rectangle (objbox.south east);
        \end{scope}
    \end{tikzpicture}
    \end{center}
    
\end{frame}

%------------------------------------------------
\begin{frame}
    \label{Test} % For the link button for the Appendix slide
    \frametitle{Sub-bullets and Links}

    \begin{itemize}
        \item You can also nest sub-bullets
              \begin{itemize}
                  \item Sub-bullet 1
                  \item Sub-bullet 2
                  \item Sub-bullet 3
                  \item Sub-bullet 4 \newline
              \end{itemize}
    \end{itemize}

    \textbf{Below is a button that links to a slide in the appendix}

    \begin{center}
        \hyperlink{Figure}{\beamergotobutton{Go to graphs}}
    \end{center}
\end{frame}

\definecolor{cardLight}{HTML}{C9DAF8} % Mã #C9DAF8 (Card 1 và 3)
\definecolor{cardDark}{HTML}{B7CFF4}  % Mã #B7CFF4 (Card 2)

%---------------------------------------------------------
%	PRESENTATION BODY SLIDES
%---------------------------------------------------------
\section{Đặt vấn đề \& mục tiêu} % Note all sections and subsections are automatically placed in your table of contents
\SectionIntro{Phần 1: Đặt vấn đề \& tổng quan}{Bối cảnh năng lượng tái tạo và thách thức trong công tác bảo trì.}

%------------------------------------------------
\begin{frame}
    \frametitle{BỐI CẢNH VÀ TÍNH CẤP THIẾT}
    \vspace{-0.2cm}
    % Fix overfull hbox by forcing exact bounding box
    \setlength{\emergencystretch}{3em}
    \begin{columns}[c,onlytextwidth]
        \begin{column}{0.42\textwidth}
            \justifying
            {\scriptsize Năng lượng mặt trời (PV) đang bùng nổ toàn cầu với tốc độ tăng trưởng \textbf{>20\%} mỗi năm (theo IEA). Tại Việt Nam, xu hướng này đang phát triển mạnh mẽ. Tính đến đầu năm 2025, điện mặt trời đã chiếm \textbf{11.85\%} tổng công suất nguồn điện cả nước.}

            \vspace{0.05cm}

            % Example box
            \begin{tikzpicture}
                \node[inner sep=3pt,rounded corners=0pt] (box) {
                    % Reduced width to 0.92 to ensure padding (inner sep) doesn't cause overflow
                    \begin{minipage}{0.92\linewidth}
                        {\scriptsize \hspace{0.1cm}\textbf{Ví dụ: NM điện mặt trời Dầu Tiếng}}
                        {\scriptsize
                        \begin{itemize}\setlength\itemsep{0pt}\setlength\parskip{0pt}
                            \item[\textcolor{green!50!black}{\ding{51}}] \textbf{Công suất:} >600 MWp.
                            \item[\textcolor{green!50!black}{\ding{51}}] \textbf{Quy mô:} Hàng trăm héc-ta.
                            \item[\textcolor{green!50!black}{\ding{51}}] \textbf{Thách thức:} Giám sát thủ công khó khăn.
                        \end{itemize}}
                    \end{minipage}
                };
                \begin{scope}[on background layer]
                    \fill[gray!15,rounded corners=0pt] ($(box.north west)+(-0.04cm,0.04cm)$) rectangle ($(box.south east)+(0.04cm,-0.04cm)$);
                \end{scope}
            \end{tikzpicture}

            \vspace{0.05cm}

            \justifying
            {\scriptsize Theo \textbf{Quy hoạch điện VIII}, dư địa phát triển năng lượng tái tạo vẫn lớn, dự kiến điện mặt trời tiếp tục \textbf{tăng trưởng mạnh đến 2030.}}
        \end{column}%
        \begin{column}{0.52\textwidth}
            \begin{center}
                \begin{tikzpicture}
                    % Use 0.99\linewidth to avoid rounding errors causing overfull hbox
                    \pgfmathsetmacro{\w}{0.99*\linewidth}
                    \pgfmathsetmacro{\h}{\w * 0.6} % Aspect ratio
                    
                    % FORCE Bounding box to be exactly the linewidth
                    % This hides the 0.75pt border spill from TeX's layout engine
                    \useasboundingbox (0,0) rectangle (\w pt, \h pt);

                    \begin{scope}
                        \clip[rounded corners=8pt] (0,0) rectangle (\w pt, \h pt);
                        \node[anchor=center,inner sep=0pt] at (0.5*\w pt, 0.5*\h pt) {\includegraphics[width=\w pt, height=\h pt, keepaspectratio=false]{images/DatVanDe/dautieng.jpg}};
                    \end{scope}
                    % Draw the border on top
                    \draw[line width=1.5pt,rounded corners=8pt] (0,0) rectangle (\w pt, \h pt);
                \end{tikzpicture}
            \end{center}
        \end{column}
    \end{columns}

\end{frame}

%------------------------------------------------

\begin{frame}
    \frametitle{TÌNH HÌNH HỆ THỐNG PIN MẶT TRỜI TRÊN VIỆT NAM}
    
    % [FIX 1] Giảm khoảng cách dưới tiêu đề (0.2 -> 0.1)
    \vspace{0.1cm} 
    
    \begin{columns}[T] 
        
        % --- CỘT TRÁI: BẢN ĐỒ ---
        \begin{column}{0.32\textwidth}
            \centering
            \vspace{0pt} 
            \includegraphics[width=0.95\textwidth]{images/DatVanDe/bandonhietVietNam.png}
        \end{column}
        
        % --- CỘT PHẢI: NỘI DUNG ---
        \begin{column}{0.65\textwidth}
            \vspace{0pt} 
            
            % === PHẦN 1: PHÁP LÝ ===
            {\Large\textbf{\textcolor{blue!70!black}{PHÁP LÝ}}}
            
            \vspace{0.05cm}
            
            \footnotesize
            \begin{itemize}
                % [FIX 5] Giảm khoảng cách giữa các item (0.1 -> 0.02)
                \setlength\itemsep{0.02cm} 
                \item Chuyển từ giai đoạn chính sách giá ưu đãi (kích cầu đầu tư ồ ạt) sang giai đoạn phát triển bền vững và có kiểm soát.
                \item Tập trung kiểm soát an toàn lưới điện và đặc biệt ưu tiên khuyến khích mô hình điện mặt trời \textbf{tự sản tự tiêu} (thay vì bán lên lưới).
            \end{itemize}
            
            % [FIX 6] Giảm khoảng cách lớn giữa 2 phần (0.4 -> 0.25)
            \vspace{0.2cm} 
            
            % === PHẦN 2: TIỀM NĂNG ===
            {\Large\textbf{\textcolor{blue!70!black}{TIỀM NĂNG}}}
            
            \vspace{0.05cm}
            
            \footnotesize
            \begin{itemize}
                \setlength\itemsep{0.02cm}
                \item \textbf{Việt Nam} nằm trong nhóm quốc gia có bức xạ mặt trời \textbf{tốt nhất Đông Nam Á}, đặc biệt tại miền Trung và Nam Bộ (bức xạ $>$4.6 kWh/kWp/ngày; $>$2500 giờ nắng/năm).
                \item \textbf{Miền Nam} phù hợp phát triển cả trang trại lớn và điện áp mái; \textbf{Miền Bắc} tuy bức xạ thấp hơn nhưng vẫn hiệu quả cho mô hình áp mái kết hợp lưu trữ.
            \end{itemize}
            
        \end{column}
    \end{columns}
\end{frame}
\begin{frame}
\label{Bungno}
	\frametitle{TÌNH HÌNH HỆ THỐNG PIN MẶT TRỜI TRÊN THẾ GIỚI}
	\begin{center}
		\includegraphics[width=0.7\textwidth]{images/TongQuan/bieudobungno.png}
	\end{center}
\end{frame}
%------------------------------------------------
\begin{frame}
    \frametitle{THÁCH THỨC}
    \vspace{0.2cm}
    
    \begin{columns}[c, onlytextwidth]
        % --- CỘT TRÁI: HÌNH ẢNH MINH HỌA ---
        \begin{column}{0.45\textwidth}
            \centering
            \begin{tikzpicture}
                % Khung bo góc tròn, chiều cao khoảng 5cm để cân đối với nội dung bên phải
                \clip [rounded corners=12pt] (0,0) rectangle (\linewidth, 5.5cm);
                % Chèn ảnh, fill đầy khung
                % Tăng width lên 1.7 và inner sep=0 để đảm bảo ảnh lấp đầy toàn bộ khung clip
                \node[anchor=center, inner sep=0] at (0.5\linewidth, 2.75cm) {
                    \includegraphics[width=1.7\linewidth]{images/TongQuan/mat.jpg}
                };
            \end{tikzpicture}
        \end{column}

        % --- CỘT PHẢI: NỘI DUNG ---
        \begin{column}{0.5\textwidth}
            % Dùng minipage với chiều cao cố định 5.5cm (bằng chiều cao ảnh bên trái)
            % để căn đều: Đầu itemize trùng đầu ảnh, Đáy box trùng đáy ảnh
            \begin{minipage}[c][5.5cm][s]{\linewidth}
                \begin{itemize}
                    \setlength\itemsep{0.4cm} 
                    \raggedright
                    \item Tốn kém quá nhiều thời gian và nhân lực.
                    \item Nguy cơ mất an toàn lao động cho kỹ thuật viên.
                    \item Hiệu quả không cao.
                \end{itemize}

                \vfill % Đẩy box xuống đáy để căn đáy với ảnh

                % Box Mục tiêu
                \begin{tcolorbox}[
                    enhanced,
                    colback=gray!10,
                    colframe=gray!40,
                    arc=8pt,
                    boxrule=1pt,
                    left=6pt, right=6pt, top=6pt, bottom=6pt,
                    width=\linewidth
                ]
                    \small
                    \textbf{Mục tiêu:} tạo ra một giải pháp cân bằng giữa độ chính xác và chi phí, giúp phát hiện sớm hư hỏng để tối ưu hóa vận hành.
                \end{tcolorbox}
            \end{minipage}
        \end{column}
    \end{columns}
    
\end{frame}

% --- ĐỊNH NGHĨA MÀU (Trích xuất từ ảnh) ---
% Màu Xanh (Box trái)
\definecolor{myBlueBorder}{HTML}{1FA2F2} % Màu thanh bên trái
\definecolor{myBlueBG}{HTML}{F0F8FF}     % Màu nền xanh nhạt

% Màu Cam (Box phải)
\definecolor{myOrangeBorder}{HTML}{F7A623} % Màu thanh bên trái
\definecolor{myOrangeBG}{HTML}{FFFBF2}     % Màu nền cam nhạt

% --- ĐỊNH NGHĨA KHUNG (Custom Box) ---
\newtcolorbox{objectivebox}[2]{
    enhanced,
    equal height group=obj,
    boxrule=0pt, frame hidden, % Ẩn viền xung quanh
    borderline west={0pt}{0pt}{#1}, % Chỉ kẻ vạch màu bên trái dày 4pt
    colback=#2, % Màu nền
    sharp corners, % Góc vuông (hoặc chỉnh arc=3pt nếu muốn bo nhẹ)
    left=5pt, right=5pt, top=10pt, bottom=10pt, % Căn lề trong
    valign=center,
    halign=center
}



\begin{frame}
    \frametitle{MỤC TIÊU NGHIÊN CỨU}

    \begin{columns}[c, onlytextwidth] % Căn lề giữa
        % --- CỘT TRÁI (Xử lý ảnh) ---
        \begin{column}{0.48\textwidth}
            \begin{objectivebox}{myBlueBorder}{myBlueBG}
                \centering
                % Đường dẫn ảnh đã được đổi dấu \ thành / để LaTeX hiểu
                \includegraphics[width=1.5cm]{images/DatVanDe/icons/xulyanh.png} 
                
                \vspace{0.5cm}
                
                \raggedright % Căn trái văn bản
                Xây dựng phương pháp \textbf{xử lý ảnh} để tự động tách chiết từng tấm pin từ ảnh toàn cảnh chụp bởi UAV.
            \end{objectivebox}
        \end{column}

        % --- CỘT PHẢI (Học sâu) ---
        \begin{column}{0.48\textwidth}
            \begin{objectivebox}{myOrangeBorder}{myOrangeBG}
                \centering
                \includegraphics[width=1.5cm]{images/DatVanDe/icons/hocsau.png}
                
                \vspace{0.5cm}
                
                \raggedright
                Phát triển mô hình \textbf{Học sâu (Deep Learning)} để phân loại trạng thái tấm pin (Bình thường vs. Lỗi) với độ chính xác cao.
            \end{objectivebox}
        \end{column}
    \end{columns}
\end{frame}

\begin{frame}
    \frametitle{PHẠM VI NGHIÊN CỨU}

    \vspace{-0.2cm}
    \begin{columns}[c]
        % --- Cột nội dung ---
        \begin{column}{0.6\textwidth}
            \begin{itemize}
                \setlength\itemsep{1.2em} 
                
                % Dữ liệu
                \item \textbf{Dữ liệu nghiên cứu:} 
                Sử dụng bộ dữ liệu ảnh màu (RGB) độ phân giải cao, được thu thập thực tế từ thiết bị bay không người lái (UAV) tại khu vực khảo sát.
        
                % Đối tượng
                \item \textbf{Đối tượng áp dụng:} 
                Nghiên cứu tập trung vào hệ thống pin mặt trời được lắp đặt tại các trang trại năng lượng mặt trời quy mô lớn.
        
                % Giới hạn
                \item \textbf{Giới hạn đề tài:} 
                Hệ thống chỉ tập trung phát hiện các dạng hư hỏng vật lý quan sát được trên bề mặt; không bao gồm các đo đạc chuyên sâu về đặc tính điện hay cấu trúc bên trong tế bào quang điện.
            \end{itemize}
        \end{column}
        
        % --- Cột hình ảnh minh họa ---
        \begin{column}{0.38\textwidth}
            \centering
            \newsavebox{\roundedfig}
            \sbox{\roundedfig}{\includegraphics[width=\linewidth]{images/DatVanDe/minhhoa.jpg}}
            \begin{tikzpicture}
                \clip[rounded corners=15pt] (0,0) rectangle (\wd\roundedfig, \ht\roundedfig);
                \node[inner sep=0pt, anchor=south west] at (0,0) {\usebox{\roundedfig}};
                % Vẽ thêm viền nếu cần
                %\draw[rounded corners=15pt, line width=1pt, gray] (0,0) rectangle (\wd\roundedfig, \ht\roundedfig);
            \end{tikzpicture}
        \end{column}
    \end{columns}

\end{frame}